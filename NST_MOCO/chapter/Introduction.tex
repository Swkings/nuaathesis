\chapter{绪论}

\section{研究背景及意义}

现实生活中,许多问题都是由相互冲突和影响的多个目标组成。人们经常会遇到在给定条件下,尽可能的使多个目标同时达到最佳的优化问题。优化问题存在的优化目标超过一个并需要同时处理,就成为了多目标优化问题 (MOPs)。当问题的变量域是有限集合时,我们称该类问题为多目标组合优化问题 (CMOPs)。多目标组合优化问题在现实世界中大量存在,它已广泛应用于大部分行业,包括但不限于运输、能源、金融和调度。由于大多数多目标组合优化问题是NP难问题,传统的确定性算法难以对这类问题在给定时间内进行求解,因此如何设计相应的多目标组合优化算法来解决这类问题,越来越受到学者们的关注,也有着重要的研究意义。

局部搜索(LS)是解决组合优化(CO)的一种有效方法,但是局部搜索的搜索空间可能很大,而且在给定的时间内,对于每一个组合,局部搜索都需要进行一次搜索,这种搜索空间可能会导致算法的运行时间很长。但是,我们不仅可以通过各种高效的搜索策略来减小局部搜索的搜索空间,而且还可以通过对具有显性结构化搜索空间的组合优化问题进行搜索空间的剪枝预处理,从而减小搜索空间,达到提高算法的运行效率的目的。

此外,迁移学习(TL)是一种利用跨问题领域的有用特征数据来提高学习性能的机器学习(ML)方法。TL利用从包含大量高质量信息的源域(SD)中获得的知识来改进只包含少量知识的目标域(TD)的学习模型 \cite{pan2009survey}。然而,现在的TL研究主要局限于机器学习应用,如计算机视觉、自然语言处理和语音识别等。近年来,在进化优化的背景下,基于TL中不同领域的信息迁移这一特征,越来越多的研究者开始将进化算法(EA)与TL结合起来,提出了进化迁移优化(ETO)这一新的范式。ETO通过将EA求解器与知识学习和跨领域信息迁移相结合,从而实现更好的优化效率和性能。

当前的研究中,很多工作都是基于启发式的方法在有限时间内来对优化问题进行优化,并给出一个或一组近似的解决方案。鉴于此,本文将在基于分解的多目标优化框架上,融入局部搜索和领域信息迁移的思想。并且,针对具体问题,设计高质量的邻域结构(NS),同时将这些邻域结构作为ETO中能够被迁移的元信息,从而使得设计的算法能够在提升优化效率的同时能够获得更好的优化效果。

\section{国内外研究进展}

从多目标组合优化问题被提出之后,国内外学者设计了大量启发式算法来对各种多目标组合优化问题进行求解,并取得十分显著的成果。其大多数是基于进化算法,可主要分为三类:基于Pareto支配关系,基于指标和基于分解的多目标优化算法:
\begin{itemize}
    \item 基于Pareto支配关系的多目标优化算法
    \item 基于指标的多目标优化算法
    \item 和基于分解的多目标优化算法
\end{itemize}

许多多目标启发式方法的一个共同点是使用局部搜索(LS)技术,这些技术既可以在现有的算法内部混合使用,也可以作为其主要组成部分。

\section{本文主要研究工作与贡献}

我

\section{本文组织结构}

我