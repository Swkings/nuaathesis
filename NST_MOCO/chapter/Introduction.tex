\chapter{绪论}

\section{研究背景及意义}

现实生活中,许多问题都是由相互冲突和影响的多个目标组成。人们经常会遇到在给定条件下,尽可能的使多个目标同时达到最佳的优化问题。优化问题存在的优化目标超过一个并需要同时处理,就成为了多目标优化问题 (MOPs)。当问题的变量域是有限集合时,我们称该类问题为多目标组合优化问题 (CMOPs)。多目标组合优化问题在现实世界中大量存在,它已广泛应用于大部分行业,包括但不限于运输、能源、金融和调度。由于大多数多目标组合优化问题是NP难问题,传统的确定性算法难以对这类问题在给定时间内进行求解,因此如何设计相应的多目标组合优化算法来解决这类问题,越来越受到学者们的关注,也有着重要的研究意义。

局部搜索(LS)是解决组合优化(CO)的一种有效方法,但是局部搜索的搜索空间可能很大,而且在给定的时间内,对于每一个组合,局部搜索都需要进行一次搜索,这种搜索空间可能会导致算法的运行时间很长。但是,我们不仅可以通过各种高效的搜索策略来减小局部搜索的搜索空间,而且还可以通过对具有显性结构化搜索空间的组合优化问题进行搜索空间的剪枝预处理,从而减小搜索空间,达到提高算法的运行效率的目的。

此外,迁移学习(TL)是一种利用跨问题领域的有用特征数据来提高学习性能的机器学习(ML)方法。TL利用从包含大量高质量信息的源域(SD)中获得的知识来改进只包含少量知识的目标域(TD)的学习模型\cite{pan2009survey}。然而,现在的TL研究主要局限于机器学习应用,如计算机视觉、自然语言处理和语音识别等。近年来,在进化优化的背景下,基于TL中不同领域的信息迁移这一特征,越来越多的研究者开始将进化算法(EA)与TL结合起来,提出了进化迁移优化(ETO)这一新的范式。ETO通过将EA求解器与知识学习和跨领域信息迁移相结合,从而实现更好的优化效率和性能。

当前的研究中,很多工作都是基于启发式的方法在有限时间内来对优化问题进行优化,并给出一个或一组近似的解决方案。鉴于此,本文将在基于分解的多目标优化框架上,融入局部搜索和领域信息迁移的思想。并且,针对具体问题,设计高质量的邻域结构(NS),同时将这些邻域结构作为ETO中能够被迁移的元信息,从而使得设计的算法能够在提升优化效率的同时能够获得更好的优化效果。

\section{国内外研究进展}

从多目标组合优化问题被提出之后,国内外学者设计了大量启发式算法来对各种多目标组合优化问题进行求解,并取得十分显著的成果。其大多数是基于进化算法,可主要分为三类:基于Pareto支配关系,基于指标和基于分解的多目标优化算法:
\begin{itemize}
    \item \textbf{基于Pareto支配关系的多目标优化算法:}这类算法的核心思想是通过支配关系和密度估计来比较解之间的优劣关系。1993年,随着Srinivas和Deb等人将Pareto支配关系融入到多目标进化算法(MOEA)中,提出了基于非支配排序的进化算法(NSGA)\cite{srinivas1994muiltiobjective},标志着多目标进化算法进入了一个新的阶段,这也是多目标进化算法框架的一个重大进展。由于NSGA存在在非支配分层上的计算复杂度高($O(N^3)$)、无精英保留策略等问题,为此,Deb等人在NSGA的基础上,于2002年提出了带精英策略的非支配排序的进化算法(NSGA-II)\cite{deb2002fast},较好的弥补了NSGA的缺点。同时,强度Pareto进化算法(SPEA2)\cite{zitzler2001spea2}也是一种性能比较好的多目标进化算法,其使用了一种外部保存集的机制来保证解的多样性。为了进一步减少非支配排序的时间复杂度,Zhang等人提出了一种新的非支配排序算法ENS\cite{zhang2014efficient},Li等人提出了一种稳态的替换策略ENLU\cite{li2016efficient}。
    \item \textbf{基于指标的多目标优化算法:}这类算法的核心思想是通过性能指标(IGD\cite{bosman2003balance},HV\cite{zitzler1999multiobjective})对解进行选择。2004年,Zitzler等人提出了基于指标的多目标进化算法,该算法主要是基于超体积指标(HV)来对个体进行选择,代表算法有IBEA\cite{zitzler2004indicator}、HypE\cite{bader2011hype}。由于涉及到HV的计算,如果问题的目标数变多时,计算复杂度也成指数增高。
    \item \textbf{和基于分解的多目标优化算法:}这类算法的核心思想是通过将一个多目标问题分解成一组单目标优化问题,然后对这些单目标问题进行求解。2007年,Zhang等人系统的将分解的思想融入到进化多目标优化算法中,提出了基于分解的多目标进化算法(MOEA/D)\cite{zhang2007moea},这一工作的出现使得分解策略被广泛的应用于多目标进化算法。近年来,MOEA/D得到广泛应用,成为了最具有影响力的MOEA之一。
\end{itemize}

许多多目标启发式方法的一个共同点是使用局部搜索(LS)技术,目前,大部分的多目标组合优化算法中使用LS来产生新的个体(解),其原理就是对现有的个体使用邻居搜索策略,从而产生一个或多个新的个体,将产生的高质量的个体替换当前的个体,以此不断迭代来逼近目标问题的最优解。LS既可以在现有的算法内部混合使用,也可以作为其主要组成部分。

进化算法(EA)是进化计算的一个子类,是基于群体搜索策略和群体中个体之间的信息交换的一类元启发式方法,通常用于解决多目标优化问题。EA通常使用迭代的方式模拟世代进化,使用交叉和变异等算子来改善种群性能(种群进化)。LS算子在多目标EA中的作用就是替代交叉和变异等算子,以实现种群的进化。Ishibuchi和Murata于1996年首次提出了多目标遗传局部搜索算法(MOGLS)\cite{ishibuchi1996multi},该算法将遗传算法和单目标的局部搜索算子进行融合,通过对每次迭代生成的个体进行局部搜索来产生新个体,Jaszkiewicz于2002年提出了另外一种遗传局部搜索算法(GLS)\cite{jaszkiewicz2002genetic},该算法使用从一组可能的权重向量中随机选择一个权重向量对个体进行局部搜索,与前者的区别在于选择个体的方式不同。Knowles和Corne提出了一种没有交叉和变异算子、仅仅依赖LS算子的Pareto解集进化策略(PAES)\cite{knowles1999pareto,knowles2000approximating}。Paquete\cite{paquete2004pareto}和Angel\cite{angel2004approximating}同时分别提出了第一个独立的局部搜索算法:Pareto局部搜索(PLS)和双标准局部搜索,这两种算法非常相似,都被统称为PLS。与EA中的种群不同,在PLS算法中,所有的个体组成的集合称为存档(Archive),并且存档集始终由Pareto解集组成。在每次迭代中,PLS算法都会从存档集中选择没有被搜索过的个体,通过邻域动作,产生其邻居解,并用这些邻居解来更新存档集。除了基于支配关系和基于分解的局部搜索算法之外,还有基于指标的多目标局部搜索算法\cite{basseur2007indicator},它使用性能指标(如HV,IGD)来选择优质的邻居个体。Drugan等人提出了PLS的多重启算法(MPLS)\cite{drugan2012stochastic}。MPLS是建立在PLS的基础之上的,MPLS中存档集中的每个个体都会被标记是否休眠,休眠状态的个体不能再次被探索。当个体被探索过后,该个体会被标记为休眠状态,在运行给定迭代次数的PLS之后,MPLS不再考虑新个体,而是从存档中随机选择一个或多个个体,并将其设置为活跃状态,重新启动。

近年来,跨领域的方法相融合的思想越来越受到学者们的关注。就如在机器学习(ML)和优化两个领域,有很多研究将迁移学习的思想运用到优化领域当中。TL是一种利用跨问题领域的知识来提高算法性能的机器学习方法,在进化优化的背景下,基于TL中不同领域的信息迁移这一特征,越来越多的研究者开始将EA与TL结合起来,提出了进化迁移优化(ETO)这一新的范式。ETO通过将EA求解器与跨领域信息迁移相结合,从而实现更好的优化效率和性能。

Feng等人于2017年提出可一种具有跨异构问题学习能力,能够自动编码进化搜索的算法框架\cite{feng2017autoencoding},该算法能够针对MOP,通过将当前的解和过去的解建立跨问题的的映射,从而将过去解中蕴含的信息转移到现在的解中,从而达到知识迁移的作用。Yang等人于2019年提出了一种用于解决MOP的ETO算法\cite{yang2019offline},该算法有两个辅助代理模型,一个是粗粒度代理模型,一个是细粒度代理模型,粗粒度代理模型旨在引导算法在整个搜索空间中快速找到一个蕴含有用信息的子区域,细粒度代理模型则侧重于利用前一个代理模型所寻找的子区域来搜索优质的解,从而达到优化问题的目的。2020年,Lin等人提出了一种寻找能够为迁移提供有效信息的、高质量的解的方法\cite{lin2020effective},在该方法中,从能够实现正迁移的解的邻居中选择可以被用来迁移的解,从而增强其解决任务的收敛性。Liang等人提出了一种两阶段自适应的知识迁移方法\cite{liang2020two},该方法通考虑了EA中MOP的种群分布,从反映整体搜索趋势的概率模型中提取知识来加速和提高任务的收敛性能,在第一转移阶段,使用自适应权重来调整个体搜索的步长,可以减少负转移的影响,在第二阶段,进一步动态调整个体的搜索范围,可以提高种群的多样性,有利于跳出局部最优。

显然,随着MOP中目标数量的增加,搜索空间呈指数增长,将被转移的信息也急剧增加\cite{tan2021evolutionary}。因此,正确的构建源域和目标域之间的映射和怎样实现源域和目标域之间的有用知识的正迁移更具挑战性。更深入地分析什么信息是有用的信息以及这些信息如何与跨问题域的多个目标相关联,可以帮助我们更好地了解ETO如何以及何时在MOP中表现良好。

\section{本文主要研究内容}

本论文主要研究使用基于分解的多目标组合优化算法解决超多目标组合优化问题。
为解决现有的基于分解的算法在超多目标分解后搜索空间剧增的问题,
我们使用局部搜索技术,设计适配问题的邻域结构,
提出了基于最小生成树和欧拉回路的邻域结构生成算法。
并且与ML领域相结合,将迁移学习中的知识迁移的技术应用于多目标组合优化算法当中,提出了基于邻域结构迁移的多目标组合优化算法。本文的主要研究内容如下:
% 当前的研究中,很多工作都是基于启发式的方法在有限时间内来对优化问题进行优化,并给出一个或一组近似的解决方案。鉴于此,本文将在基于分解的多目标优化框架上,融入局部搜索和领域信息迁移的思想。并且,针对具体问题,设计高质量的邻域结构(NS),同时将这些邻域结构作为ETO中能够被迁移的元信息,从而使得设计的算法能够在提升优化效率的同时能够获得更好的优化效果。

% 从问题本身出发,挖掘问题本身的信息,构建邻居结构,图优化问题,TSP,VRP等,
\begin{enumerate}
    \item 本论文基于分解的多目标优化框架,使用局部搜索技术,针对具体问题,从问题本身出发,挖掘问题本身的信息,构建邻域结构,提出了相应的邻域结构生成算法,该算法能够快速地为路径类问题(TSP,VRP等)生成高质量的邻域结构,供局部搜索使用,从而提升问题的优化效率。
    \item 基于上述邻居结构,结合ETO中信息迁移的思想,提出了基于邻域结构迁移的多目标组合优化算法。两目标相似度模型粗粒度地确定能被迁移的邻域结构的范围,勘探-开发机制细粒度地动态改变邻域结构被选择用来迁移的概率,从而提高在算法中发生正迁移的比重,以达到更好的收敛效果和优化质量。
\end{enumerate}

\section{本文组织结构}

本论文围绕基于分解的多目标优化框架和用于局部搜索的邻域结构,以及基于邻域结构迁移的多目标组合优化算法进行研究。通过分析现有算法的不足,并且针对具有不同规模、类型和特征的组合优化问题,进行算法设计及实验论证分析。同时,通过对比现有的相关算法来验证设计的算法的合理性及有效性。本论文总共分为五章节,全文组织结构如下:

第一章首先介绍了本文的研究背景及意义,梳理了多目标优化算法,局部搜索和进化迁移等邻域的相关发展历程,最后阐述了本文的主要研究内容及组织结构。

第二章首先介绍了多目标优化问题的定义及相关概念,并且对一些著名的多目标进化算法进行了详细的阐述,然后性能评价指标,邻域结构,进化迁移以及常见的测试问题进行了描述,为后续章节提供基础理论支撑。

第三章提出了邻域结构生成算法

第四章提出了基于邻域结构迁移的多目标进化算法

第五章对本文的主要工作进行总结,并分析本文工作的不足,为后续研究工作提供进一步可能的研究方向