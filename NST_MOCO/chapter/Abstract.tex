% 摘要
\begin{abstract}
\label{abstract}
    多目标组合优化问题在现实世界中大量存在,它已广泛应用于大部分行业,包括但不限于运输、能源、金融和调度,设计相应的多目标组合优化算法来解决这类问题具有重要的实际意义。本文围绕组合优化问题和基于分解的多目标组合优化算法进行研究,主要研究内容包括以下两个部分:
    \par
    第一,局部搜索技术是解决组合优化问题的一种有效方法,但由于组合优化问题拥有巨大的搜索空间会导致局部搜索算法的搜索时间过长、易陷入局部最优。针对这一问题,我们不仅可以通过各种高效的搜索策略来减小局部搜索的搜索空间,而且还可以通过对具有显性结构化搜索空间的组合优化问题进行搜索空间的剪枝预处理,从而减小搜索空间,达到提高算法的运行效率的目的。本文就研究的组合优化问题,设计适配问题的邻域结构,提出了基于最小生成树和欧拉回路的邻域结构生成算法,并通过局部搜索技术对这些邻域结构进行搜索,以此提升问题的优化效率。通过实验对比,可以证实基于最小生成树和欧拉回路的邻域结构生成算法生成邻域结构的高效率及其邻域结构的高质量。
    \par
    第二,基于适配于组合优化问题的邻域结构,结合进化迁移中不同领域信息迁移这一特征,提出了基于邻域结构迁移的多目标组合优化算法,并在邻域结构迁移中构建了两目标相似度模型和多臂老虎机模型。两目标相似度模型粗粒度地确定能被迁移的邻域结构的范围,多臂老虎机模型细粒度地学习关于选择邻域结构的知识,以便算法能够获得更大的收益。最后,通过实验能够表明基于邻域结构迁移的多目标组合优化算法在各种测试用例上获得的非支配解集不仅分布性好,而且收敛性也优异,并且验证了邻域结构迁移中的两目标相似度模型和多臂老虎机模型的有效性。
\end{abstract}
\keywords{多目标优化, 局部搜索, 进化迁移优化, 邻域结构, 最小生成树, 邻域结构迁移}

\begin{abstractEn}
\label{abstractEn}
    In the real world, there are a large number of multi-objective combinatorial optimization problems that are widely used in most industries, including but not limited to transportation, energy, finance, and dispatch. It is of great practical significance to design corresponding multi-object combinatorial optimization algorithms to solve such problems. This thesis focuses on combinatorial optimization problems and decomposition-based multi-objective combinatorial optimization algorithms. The main research content of this thesis includes the following two parts:
    \par
    First, the local search technology is an effective method to solve the combinatorial optimization problem. However, because the combinatorial optimization problem has a huge search space, the search time of the local search algorithm is too long, and it is easy to fall into the local optimum. In response to this problem, we can not only reduce the search space of local search through a variety of efficient search strategies but also perform search space pruning preprocessing for combinatorial optimization problems with explicitly structured search spaces, thereby reducing the search space to achieve the goal of improving the operating efficiency of the algorithm. This thesis focuses on combinatorial optimization problems, designs the neighbor structure of the adaptation problem, proposes a neighbor structure generation algorithm based on minimum spanning tree and Euler circuit, and searches these neighbor structures through local search technology to improve The optimization efficiency of the problem. Through experimental comparison, It can be confirmed by experimental comparison that the neighbor structure generation algorithm based on the minimum spanning tree and Euler circuit has the high efficiency of generating neighbor structure and the high quality of the neighbor structure.
    \par
    Second, based on the neighbor structure adapted to the combinatorial optimization problem, combined with the characteristics of knowledge learning and transfers across related domains in evolutionary transfer optimization, a multi-objective combinatorial optimization algorithm based on neighbor structure transfer is proposed, and the neighbor structure transfer is composed of the bi-objective similarity model and the multi-armed bandit model. The bi-objective similarity model determines the neighbor structure set that can be transferred in a coarse-grained manner, and the multi-armed bandit model learns the knowledge about the selected neighbor structure in a fine-grained manner so that the algorithm can obtain greater benefits. Finally, through experiments, it can be shown that the non-dominated solution set obtained by the multi-objective combinatorial optimization algorithm based on neighbor structure transfer in various test cases is not only good in distribution, but also excellent in convergence, and verifies the effectiveness of the bi-objective similarity model and multi-armed bandit model in the neighbor structure transfer.
\end{abstractEn}
\keywordsEn{Multi-objective Optimization, Local Search, Evolutionary Transfer Optimization, Neighbor Structure, Minimum Spanning Tree, Neighborhood Structure Transfer}