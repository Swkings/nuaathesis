\chapter{基于最小生成树和欧拉回路的邻域结构生成方法}
\label{chap:NS_Method}

\section{引言}
\label{sec:NS_Method:引言}
% 需要用LS中的NS,从何而来,为什么用,如何用,最终结果如何?
Fourman等人于1985年在解决多目标优化问题时提出多目标进化算法(MOEA)\cite{fourman1985compaction,schaffer1985multiple},其实质就是进化算法(EA)在多目标优化领域的一种扩展。由章节\ref{sec:背景介绍:多目标组合优化算法}~介绍可知,EA是一种受达尔文进化论启发而产生的、基于种群且不断迭代进化的一大类元启发式算法,其中进化规则中的遗传操作正是EA的核心部分。EA通过模拟自然界物种适者生存、优胜劣汰的遗传机制,从而使得种群在表现出基因多样性的同时相互竞争,以致整个种群得以进化。而遗传机制其中一个特征就是,通过复制、交叉、变异操作从父代种群个体的基因特征中产生新的子代个体。然而对于大多数多目标组合优化问题而言,其决策空间(基因编码形式)是离散并十分巨大的,导致了这些遗传操作(复制、交叉、变异算子等)生成子代个体的基因特征变化不大,生成新个体的质量和效率都会大大降低,从而使得种群的进化较为缓慢。因此,我们需要采用一些策略或算法来替代遗传操作的部分甚至全部功能,嵌入到多目标进化算法当中,从而改善生成新个体质量和效率低的问题。
\par
通过章节\ref{sec:背景介绍:局部搜索}~的介绍可知,局部搜索算子是能够改善甚至替代多目标进化算法中遗传操作来生成新个体的方法之一。对于一个组合优化问题而言,局部搜索算子就是在该问题的候选解空间上进行搜索的一种方法,它依赖邻域结构来生成高质量的邻居解(新个体),这正是遗传操作的特征之一。而邻域结构正是局部搜索算法中最核心的部分之一。一个质量好的邻域结构,不仅能够能够帮助局部搜索算子搜索到高质量的解,而且能够加速整个算法的运行速度,使得MOEA收敛得更好且更快。比如,Lust等人\cite{lust2010speed}采用非支配排序的方式为二目标TSP问题构建了一个稀疏图(邻域结构),从而加速2PPLS\cite{lust2010two}算法求解该问题的运行速度。并且,LKH(Lin Kernighan Helsgaun)\cite{helsgaun2000effective}是目前求解TSP问题最有效的方法之一,其在算法中也用到了名为$\alpha$-nearest\cite{held1970traveling,held1971traveling}的方法生成的邻域结构,使用该邻域结构的LKH不仅提高了算法整体的运行效率,同时,算法中更是采用了一些特殊的规则与邻域结构相配合,使得算法能够快速得搜索到高质量的候选解。这些足以说明邻域结构在算法中能够发挥的作用。
\par
邻域结构除了能够作为局部搜索算子的一部分,还能够表征对应问题的内在信息被用来在不同域(SD,TD)中迁移的知识。由章节\ref{sec:背景介绍:进化迁移优化}~可知,在MOPs中,目前进化迁移优化(ETO)都是将目标空间的解(非支配解,非最优解)当作被迁移的知识。在本文中,我们将邻域结构当作被迁移的知识,提出一种基于邻域结构迁移的多目标组合优化算法。
\par
然而,邻域结构的构建与我们所要解决的问题紧密相关。邻域结构展现的形式需要我们对具体的组合优化问题有深刻的理解,并且能将该问题的内在信息用某种结构表示出来。这也正是我们所言研究的重点。我们将通过本章节对TSP问题的邻域结构的生成方法进行详细地介绍,并期望研究者能够从该工作中得到启发,设计出更具泛化性(适用于多种组合优化问题)的邻域结构。

\section{研究动机}
\label{sec:NS_Method:研究动机}
% 现在研究的工作,现有工作的缺点和能够借鉴的地方,我的工作

\section{算法框架}
\label{sec:NS_Method:算法框架}

\section{实验结果与分析}
\label{sec:NS_Method:实验结果与分析}

\section{本章小结}
\label{sec:NS_Method:本章小结}