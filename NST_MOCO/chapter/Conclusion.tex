\chapter{研究工作总结和展望}
\label{chap:conclusion}
\section{本文总结}
\label{sec:conclusion}
多目标优化问题在现实生活中随处可见,且在实际工程问题中大多是多目标组合优化问题,这些问题广泛地存在各行业中,因此,对这类问题进行优化求解有利于提高各行业的生产效率,具有重要的现实意义。然而,对于多目标组合优化问题而言,求解这类问题的精确算法的复杂度极高,因此往往采用启发式的近似算法来逼近最优解的方式来对问题进行求解,并且这些启发式的算法大多是基于种群进化框架的。又由于进化算法的易扩展性,很容易与其他领域的知识相结合,从而提升算法的整体性能。比如在进化算法中嵌入局部搜索,迁移学习等。局部搜索也是解决组合优化问题的一种有效方法,但是局部搜索的搜索效率取决于问题的搜索空间的大小,在给定时间内,对于问题解的每一个组合,局部搜索都需要进行搜索,这会影响算法的搜索效率。综上原因,本文在基于分解的多目标进化算法的基础上,结合了局部搜索中为问题本身构建的邻域结构和进化迁移优化中的跨领域信息迁移,提出了基于邻域结构迁移的多目标组合优化算法。
\par
本文从多目标问题分解后搜索空间剧增的问题出发,设计适配问题的邻域结构来降低局部搜索的搜索空间,并且与机器学习领域相结合,将迁移学习中的知识迁移的技术应用于多目标组合优化算法中,以此从问题和算法两方面来提高算法的整体性能。下面对本文各章节内容进行总结。
\begin{enumerate}
    \item 第二章首先介绍了多目标组合优化问题的背景知识,并着重对基于分解的多目标进化算法进行了介绍。然后介绍了常用在优化算法中的局部搜索技术,并且详细阐述了局部搜索中的两个概念:邻域结构和邻域动作。接着介绍了进化迁移优化跨领域信息迁移的基本思想。最后介绍了几种度量算法性能的评价指标,并给出几个测试用例的说明和定义。
    \item 第三章首先介绍了进化算法中的遗传操作用在大规模组合优化问题上的一些不足,从而引出用局部搜索来替代遗传操作在算法中的作用。然后介绍了研究局部搜索中的邻域结构的动机。接着介绍了一些在邻域结构生成算法中用到的概念和定义。之后,提出了基于最小生成树和欧拉回路的邻域结构(meNS)的生成算法,并对该算法的各部分进行了详细的介绍。最后通过与其他算法的对比,证明了meNS生成算法的高效率和最终生成的邻域结构的高质量。
    \item 第四章首先介绍了现有的多目标组合优化算法的基本思想、优点及可改进的地方,并通过第三章中对邻域结构的研究,结合进化迁移优化,引出了本章对邻域结构迁移的研究动机。然后围绕邻域结构迁移,提出了基于邻域结构迁移的多目标组合优化算法(NST-MOEA),并对邻域结构迁移中的两目标相似度模型和多臂老虎机模型进行了详细的阐述。最后通过与其他算法的对比,能够证明NST-MOEA在各种测试用例上获得的非支配解集不仅分布性好,而且收敛性也优异,同时也通过实验证实了邻域结构迁移中的两目标相似度模型和多臂老虎机模型的有效性。
\end{enumerate}

\section{研究展望}
\label{sec:future}
本论文主要研究使用基于分解的多目标组合优化算法解决多目标组合优化问题,并且针对组合优化问题和多目标组合优化算法两个方面分别提出了基于最小生成树和欧拉回路的邻域结构(meNS)的生成算法,以及基于邻域结构迁移的多目标组合优化算法(NST-MOEA)。
\par
通过实验可以证明这两种算法表现的优异性,但同样也有一些局限性。在meNS生成算法中,邻域结构和所求的组合优化问题密切相关,且邻域结构从某种程度来说可以是一个问题的具象形式。这就要求邻域结构和所求的组合优化问题能够表现出一致性。但是,实际生活中的组合优化问题往往难以确切的给出一个能够形式化表达自身的邻域结构,这就导致了meNS生成算法只能应对某一类或某一种组合优化问题。在NST-MOEA中也存在同样的局限性,因为NST-MOEA中是基于邻域结构迁移的,邻域结构仍然是算法中的关键部件,这也就导致了NST-MOEA也只能应对某一类或某一种多目标组合优化问题。
\par
虽然邻域结构是针对具体问题而提出的,有着很大的局限性,但是这种挖掘问题本身信息结构的思想可以应用于其他组合优化问题,并且可以用其他形式的信息或结构来表达问题,并将这些信息或结构作为知识迁移的一部分嵌入到多目标组合优化算法中。因此,突破邻域结构的局限性,研究组合优化问题更加通用的信息表达形式,以此设计出扩展性更好的基于迁移的多目标组合优化算法,是我们未来的研究方向之一。