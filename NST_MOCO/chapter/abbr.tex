% 注释表和缩略词,硕博论文用。
% 《要求》没有规定内容格式,按照自己的喜好来改吧。
% 注意,表格里的文字不要太长哦。

\chapter*{注释表}
\label{chap:注释表}

\noindent\begin{longtabu} to \textwidth {|X[l]|p{4.5cm}|X[l]|p{4.5cm}|}\hline
$O$ & 时间(空间)复杂度 & $m$ & 目标数量 \\ \hline
$F(\mathbf{x})$ & $\mathbf{x}$对应的目标函数 & $f_i(\mathbf{x})$ & 第$i$个目标函数 \\ \hline
$\Omega$ & 决策空间 & $\mathbb{R}^n$ & $n$维实数空间 \\ \hline
$\prec$ & 支配(强支配) & $\not \prec$ & 非支配 \\ \hline
$\preceq$ & 弱支配 & $\mathbf{x}^*$ & Pareto最优解  \\ \hline 
$PS$ & Pareto最优解集 & $PF$ & Pareto前沿\\ \hline 
$\mathbf{z}^*$ & 理想点,参考点 & $\mathbf{z}^{nad}$ & 边界点 \\ \hline
$\boldsymbol{\lambda}$ & 权重向量 & $\boldsymbol{\lambda}^i$ & 第$i$权重向量 \\ \hline
$\lambda^i_j$ & 第$i$个权重向量的第$j$分量 & $g^{ws}(x|\lambda_i)$ & $x$对应$\lambda_i$的加权值 \\ \hline
$g^{tch}(x|\lambda_i)$ & $x$对应$\lambda_i$的切比雪夫值 & $g^{pbi}(x|\lambda_i)$ & $x$对应$\lambda_i$的PBI值 \\ \hline
$B(i)$ & 第$i$个子问题的邻居 & $\mathcal{EP}$ & 外部集(精英种群) \\ \hline
$\mathcal{SP}$ & 子问题 & $\mathcal{S}$ & 搜索(解)空间 \\ \hline
$\mathcal{N}:\mathcal{S}\rightarrow 2^{\mathcal{S}}$ & 邻域动作 & $\mathcal{N}(s)$ & 解$s$的邻域 \\ \hline
$\mathcal{X}$ & 有限(可数无限)变量集 & $s^*$ & 局部最优解 \\ \hline
$\mathcal{D}$ & 决策域 & $\mathcal{C}$ & 约束域 \\ \hline
$\mathcal{A}$ & 接受策略 & $\mathcal{N}$ & 邻域结构 \\ \hline
$d(a,b)$ & 向量$a$与向量$b$之间的欧氏距离 & $C(A,B)$ & 集合$B$被集合$A$所支配的程度 \\ \hline
$\mathcal{G} = (V,E,W)$ & 带权无向图 & $V$ & 带权无向图中的点集 \\ \hline
$E$ & 带权无向图中的边集 & $W$ & 带权无向图中的边权集 \\ \hline
$\alpha$ & $\alpha-nearest$边权评估值 & $\pi$ & 边权惩罚值 \\ \hline
$\sigma$ & 次梯度优化步长 & $M^i$ & 第$i$个子问题MAB模型参数 \\ \hline
$t^i_j$ & 第$i$个子问题中第$j$个子问题被选择的次数 & $\mu^i_j$ & 第$i$个子问题中第$j$个子问题被选择的奖励均值 \\ \hline

\end{longtabu}

\chapter*{缩略词}
\label{chap:缩略词}

\noindent\begin{longtabu} to \textwidth {|X[1,c]|X[4,c]|}\hline
缩略词 & 英文全称 \\ \hline
MOPs & Multi-objective Optimization Problems \\ \hline
CMOPs & Combinatorial Multi-objective Optimization Problems \\ \hline
LS & Local Search \\ \hline
CO & Combinatorial Optimization \\ \hline
TL & Transfer Learning \\ \hline
ML & Machine Learning \\ \hline
SD & Source Domains \\ \hline
TD & Target Domains \\ \hline
EA & Evolutionary Algorithm \\ \hline
ETO & Evolutionary Transfer Optimization \\ \hline
NS & Neighborhood Structure \\ \hline
MOEA & Multi-objective Evolutionary Algorithm \\ \hline
NSGA & Non-dominated Sorting Genetic Algorithm \\ \hline
NSGA-II & A Fast and Elitist Multi-objective Genetic Algorithm-II \\ \hline
SPEA & Strength Pareto Evolutionary Algorithm \\ \hline
ENS & Efficient Approach to Non-dominated Sorting \\ \hline
ENLU & Efficient Non-domination Level Update Method \\ \hline
IGD & Inverted Generational Distance \\ \hline
HV & Hypervolume \\ \hline
IBEA & The Indicator-based Evolutionary Algorithm \\ \hline
HypE & Hypervolume Estimation Algorithm \\ \hline
MOEA/D & Multi-objective Evolutionary Algorithm based on Decomposition \\ \hline
MOGLS & Multi-objective Genetic Local Search \\ \hline
GLS & Genetic Local Search \\ \hline
PAES & Pareto Archived Evolution Strategy \\ \hline
PLS & Pareto Local Search \\ \hline
MPLS & Multi-restart Pareto Local Search \\ \hline
TSP & Traveling Salesman Problem \\ \hline
VRP & Vehicle Routing Problem \\ \hline
NDSet & Non-dominated Solution Set \\ \hline
PS & Pareto Set \\ \hline
PF & Pareto Front \\ \hline
NDSort & Non-dominated Sort \\ \hline
WS & Weight Sum \\ \hline
TCH & Tchebycheff \\ \hline
PBI & Penalty-based Boundary Intersection \\ \hline
EP & External(Elite) Population \\ \hline
SP & Sub-problem \\ \hline
GS & Global Search \\ \hline
MOTSP & Multi-Objective Traveling Salesman Problem \\ \hline
CSet & Candidate Set \\ \hline
IGD & Inverted Generational Distance \\ \hline
MCMST & Multi-criteria Minimum Spanning Tree \\ \hline
MST & Minimum Spanning Tree \\ \hline
LKH & Lin Kernighan Helsgaun \\ \hline
KNN & K-Nearest Neighbor \\ \hline
1MST & Minimum Spanning 1-tree \\ \hline
meNS & Neighborhood Structure based on Minimum spanning tree and Euler circuit \\ \hline
NST & Neighbor Structure Transfer \\ \hline
IG & Information Gain \\ \hline
MAB & Multi-armed Bandit \\ \hline
UCB & Upper Confidence Bound \\ \hline
EP & External Population \\ \hline
BSM & Bi-objective Similarity Model \\ \hline


\end{longtabu}
