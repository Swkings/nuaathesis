\iffalse
  % 本块代码被上方的 iffalse 注释掉,如需使用,请改为 iftrue
  % 使用 Noto 字体替换中文宋体、黑体
  \setCJKfamilyfont{\CJKrmdefault}[BoldFont=Noto Serif CJK SC Bold]{Noto Serif CJK SC}
  \renewcommand\songti{\CJKfamily{\CJKrmdefault}}
  \setCJKfamilyfont{\CJKsfdefault}[BoldFont=Noto Sans CJK SC Bold]{Noto Sans CJK SC Medium}
  \renewcommand\heiti{\CJKfamily{\CJKsfdefault}}
\fi

\iffalse
  % 本块代码被上方的 iffalse 注释掉,如需使用,请改为 iftrue
  % 在 XeLaTeX + ctexbook 环境下使用 Noto 日文字体
  \setCJKfamilyfont{mc}[BoldFont=Noto Serif CJK JP Bold]{Noto Serif CJK JP}
  \newcommand\mcfamily{\CJKfamily{mc}}
  \setCJKfamilyfont{gt}[BoldFont=Noto Sans CJK JP Bold]{Noto Sans CJK JP}
  \newcommand\gtfamily{\CJKfamily{gt}}
\fi


% 设置基本文档信息,\linebreak 前面不要有空格,否则在无需换行的场合,中文之间的空格无法消除
\nuaaset{
  thesisid = {1028716 22-S059},   % 论文编号
  % title = {\nuaathesis{} 快速上手\linebreak 示例文档},
  title = {基于邻域结构迁移的多目标组合优化算法的研究},
  author = {王康},
  college = {计算机科学与技术 学院},
  advisers = {蔡昕烨\quad 副教授},
  % applydate = {二〇一八年六月}  % 默认当前日期
  %
  % 本科
  major = {计算机科学与技术},
  studentid = {131810299},
  classid = {应用技术},           % 班级的名称
  industrialadvisers = {Jack Ma}, % 企业导师,若无请删除或注释本行
  % 硕/博士
  majorsubject = {计算机科学与技术},
  researchfield = {计算智能理论与应用},
  libraryclassid = {TP371},       % 中图分类号
  subjectclassid = {080605},      % 学科分类号
}
\nuaasetEn{
  title = {Research on Multi-Objective Combination Optimization Algorithm Based on Neighbor Structure Transfer},
  author = {Wang Kang},
  college = {College of Computer Science and Technology},
  majorsubject = {Computer Science and Technology},
  advisers = {Associate Prof.~Cai Xinye},
  degreefull = {Master of Engineering},
  % applydate = {June, 8012}
}

% 摘要
% 摘要
\begin{abstract}
    基于邻域结构迁移的多目标组合优化算法的研究。
\end{abstract}
\keywords{多目标优化, 局部搜索, 进化迁移优化, 邻域结构, 最小生成树, 邻域结构迁移}

\begin{abstractEn}
    Research on Multi-Objective Combination Optimization Algorithm Based on Neighbor Structure Transfer.
\end{abstractEn}
\keywordsEn{Multi-objective Optimization, Local Search, Evolutionary Transfer Optimization, Neighborhood Structure, Minimum Spanning Tree, Neighborhood Structure Transfer}

% 请按自己的论文排版需求,随意修改以下全局设置

\usepackage{subfig}
\usepackage{rotating}
\usepackage[usenames,dvipsnames]{xcolor}
\usepackage{tikz}
\usepackage{pgfplots}
\pgfplotsset{compat=1.16}
\pgfplotsset{
  table/search path={./fig/},
}
\usepackage{ifthen}
\usepackage{longtable}
\usepackage{siunitx}
\usepackage{listings}
\usepackage{multirow}
\usepackage{pifont}

% \usepackage[ruled,linesnumbered]{algorithm2e}
% 按章节编号
% vlined 去掉算法中的封闭语法end
\usepackage[ruled,linesnumbered,resetcount,algochapter]{algorithm2e}
\usepackage{setspace}
% 三线表,表格注释说明
\usepackage{threeparttable}

\lstdefinestyle{lstStyleBase}{%
  basicstyle=\small\ttfamily,
  aboveskip=\medskipamount,
  belowskip=\medskipamount,
  lineskip=0pt,
  boxpos=c,
  showlines=false,
  extendedchars=true,
  upquote=true,
  tabsize=2,
  showtabs=false,
  showspaces=false,
  showstringspaces=false,
  numbers=left,
  numberstyle=\footnotesize,
  linewidth=\linewidth,
  xleftmargin=\parindent,
  xrightmargin=0pt,
  resetmargins=false,
  breaklines=true,
  breakatwhitespace=false,
  breakindent=0pt,
  breakautoindent=true,
  columns=flexible,
  keepspaces=true,
  framesep=3pt,
  rulesep=2pt,
  framerule=1pt,
  backgroundcolor=\color{gray!5},
  stringstyle=\color{green!40!black!100},
  keywordstyle=\bfseries\color{blue!50!black},
  commentstyle=\slshape\color{black!60}}

%\usetikzlibrary{external}
%\tikzexternalize % activate!

\newcommand\cs[1]{\texttt{\textbackslash#1}}
\newcommand\pkg[1]{\texttt{#1}\textsuperscript{PKG}}
\newcommand\env[1]{\texttt{#1}}

\theoremstyle{nuaaplain}
\nuaatheoremchapu{definition}{定义}
\nuaatheoremchapu{theorem}{定理}
\nuaatheoremchapu{corollary}{推论}
\nuaatheoremchapu{assumption}{假设}
\nuaatheoremchap{exercise}{练习}
\nuaatheoremchap{nonsense}{胡诌}
\nuaatheoremg[句]{lines}{句子}

\renewcommand{\algorithmcfname}{算法}
