局部搜索(Local Search)是求解组合优化问题的一种经典的启发式算法。局部搜索算
法的优点是简单、灵活且易于实现,然而容易使算法陷入局部最优并且常常出现循环现
象。跳出局部最优的机制对局部搜索而言非常的关键,目前常见的跳出局部最优的机制:
随机重启策略、随机游走策略、随机扰动策略等。循环现象是指在搜索过程中算法重复
地访问一个已经访问过的候选解,这一现象不仅浪费时间还容易使算法陷入局部最优。
为了避免这一现象,目前已经有一些策略被提出,如禁忌策略和格局检测策略。精确算
法能够求出问题的最优解,但是随着问题规模的增大,精确算法往往不能在可接受的时
间内找到最优解。而局部搜索算法能够在可接受的时间内找到一个最优解或近似最优解。
局部搜索算法从一个初始解开始,通过邻域动作,产生其邻居解,通过判断邻居解的质
量,根据某种启发式策略,继而选择邻居解,重复上述过程,直至达到终止条件。局部
搜索算法有两个阶段:初始化阶段和局部搜索阶段。首先,算法构造一个初始解并将最
优解初始化为初始解。在搜索过程中,每次从当前候选解的所有邻居候选解中选择一个
邻居解,然后更新当前候选解和最优解。最后算法返回最优解。